\documentclass{lab}

\usepackage[utf8]{inputenc}
\usepackage{graphicx}
\usepackage{tocloft}
\usepackage{amssymb}
\usepackage{xcolor}
\usepackage{enumitem}
\usepackage{amsmath} 
\usepackage{animate}
\usepackage{graphicx}
\usepackage[T1]{fontenc} 
\usepackage[top=0.5in, bottom=0.5in, left=0.5in, right=0.5in]{geometry}
\usepackage{mathrsfs}
\usepackage{hyperref}
\usepackage{wrapfig}
\usepackage{indentfirst}
\usepackage{mathtools}
\usepackage[export]{adjustbox}
\usepackage{scrextend}
\usepackage{animate}
\usepackage{float}
\newcommand\tab[1][1cm]{\hspace*{#1}}
\geometry{
    margin=0.5in,
    left=2.2cm,
    right=2.2cm,
    headheight=17pt,
    headsep=1cm, % Increase the margin from the top for the header
    includehead,
    includefoot
}

% python color
\usepackage{listings}
\usepackage{xcolor}
\usepackage{caption}

% Define VSC Dark+ colors
\definecolor{vscode-background}{rgb}{0.0, 0.0, 0.0}
\definecolor{vscode-foreground}{rgb}{0.96, 0.96, 0.96}
\definecolor{vscode-comment}{rgb}{0.51, 0.51, 0.51}
\definecolor{vscode-orange}{rgb}{1.0, 0.6, 0.0}
\definecolor{vscode-yellow}{rgb}{0.98, 0.83, 0.26}
\definecolor{vscode-green}{rgb}{0.19, 0.8, 0.35}
\definecolor{vscode-cyan}{rgb}{0.16, 0.71, 0.73}
\definecolor{vscode-blue}{rgb}{0.13, 0.48, 0.78}
\definecolor{vscode-purple}{rgb}{0.58, 0.4, 0.72}
\definecolor{vscode-pink}{rgb}{0.8, 0.36, 0.36}

% Define the style
\lstdefinestyle{vscode-darkplus-python}{
    language=Python,
    backgroundcolor=\color{vscode-background},
    basicstyle=\color{vscode-foreground}\ttfamily\small,
    commentstyle=\color{vscode-comment},
    keywordstyle=\color{vscode-orange},
    numberstyle=\tiny\color{vscode-comment},
    numbers=left,
    stringstyle=\color{vscode-green},
    emphstyle=\color{vscode-pink},
    frame=none,
}
% Change the caption label to "Kod źródłowy"
\DeclareCaptionLabelFormat{myformat}{Kod źródłowy #2}

\usepackage[T1]{fontenc}
\usepackage[polish]{babel}
\usepackage[utf8]{inputenc}
\usepackage{amsmath}
\usepackage{listings}
\usepackage{hyperref}
\usepackage{pythonhighlight}
\usepackage{amssymb}
\usepackage{mathtools}
\usepackage{array}
\usepackage{xpatch}

\xpretocmd{\part}{\setcounter{section}{0}}{}{}
\usepackage[margin=0.5in,
    left=2.2cm,
    right=2.2cm,
    headheight=17pt,
    includehead, includefoot]{geometry}
    
%% nagłówek
\usepackage{fancyhdr}
\usepackage{tikz}
\usepackage{pgfplots}
\usepackage{pgfplotstable}
\pagestyle{fancy}
\fancyhf{}
\rhead{Rachunek Prawdopodobieństwa i Statystyka}
\lhead{Porównanie Algorytmów Minimalizacji Stochastycznej}
\rfoot{\thepage}

%% Podmiana \part na obecne wyświetlanie
\makeatletter
\renewcommand\@part[2][]{%
  \ifnum \c@secnumdepth >-2\relax
    \refstepcounter{part}%
    \addcontentsline{toc}{part}{\thepart#2}%
  \else
    \addcontentsline{toc}{part}{#2}%
  \fi
  \markboth{}{}%
  {\raggedright % Align title to the left
   \interlinepenalty \@M
   \normalfont
   \ifnum \c@secnumdepth >-2\relax
     \Large\bfseries \thepart\hspace{1em}%
   \fi
   \huge \bfseries #2\par}%
  \@endpart
}
\makeatother

\renewcommand\thepart{\Roman{part}. }

\addtokomafont{labelinglabel}{\sffamily}

\renewcommand{\cftsecleader}{\cftdotfill{\cftdotsep}}
\renewcommand{\cftsubsecleader}{\cftdotfill{\cftdotsep}}


\begin{document}
\captionsetup[lstlisting]{labelformat=myformat}

\begin{figure*}
    \centering
    \includegraphics{img/agh.png}
\end{figure*}
\title{\Huge \textbf{Porównanie Algorytmów Minimalizacji Stochastycznej}\\ Pure-Random-Search i Multi-Start}
\author{Radosław Rolka, Mateusz Kochelski\\Rachunek Prawdopodobieństwa i Statystyka\\Informatyka WI AGH, II rok}
\date{styczeń 2023}

\maketitle
\newpage
\tableofcontents
\thispagestyle{fancy} 
\newpage

\part{Wstęp}
\section{Wybrane Algorytmy Minimalizacji}
\subsection{Pure-Random-Search}
TODO

\subsection{Multi-Start}
TODO

\section{Wybrane Funkcje}
\subsection{Alpine01}
TODO

\subsection{funkcja Ackley’a}
TODO

\newpage
\part{Metodologia}
\section{Sposób wykonania}
TODO

\section{Kod źródłowy}
\subsection{Pure-Random-Search}
TODO

\subsection{Multi-Start}
\begin{lstlisting}[
  style=vscode-darkplus-python,
  caption={Funkcja wykorzystująca metodę MS.},
  label=lst:python_code,
  captionpos=b
]
MS_Minimalize <- function(func, dim, points_quantity) {
  f <- func(dim)
  values <- c(1:points_quantity)
  counter <- c(1:points_quantity) 

  MS <- replicate(n = points_quantity, optim(runif(dim, 
                                                   getLowerBoxConstraints(f), 
                                                   getUpperBoxConstraints(f)), 
                                             f, 
                                             method = "L-BFGS-B",
                                             lower = getLowerBoxConstraints(f), 
                                             upper = getUpperBoxConstraints(f))
  )
  for (i in 1:points_quantity) {
    values[i]  = MS[[2, i]]
    counter[i] = MS[[3, i]][2] 
  }
  return (list(min(values), mean(counter)))
}
\end{lstlisting}

\subsection{Porównanie metod}
TODO

\subsection{}

\newpage
\part{Porównanie}
\section{dim = 2}
\subsection{Wykresy Alpine01}
\subsubsection{Pure-Random-Search}
TODO

\subsubsection{Multi-Start}
TODO

\subsubsection{Obserwacje}
TODO

\subsection{Wykresy funkcji Ackley’a}
\subsubsection{Pure-Random-Search}
TODO

\subsubsection{Multi-Start}
TODO

\subsubsection{Obserwacje}
TODO

\section{dim = 10}
\subsection{Wykresy Alpine01}
\subsubsection{Pure-Random-Search}
TODO

\subsubsection{Multi-Start}
TODO

\subsubsection{Obserwacje}
TODO

\subsection{Wykresy funkcji Ackley’a}
\subsubsection{Pure-Random-Search}
TODO

\subsubsection{Multi-Start}
TODO

\subsubsection{Obserwacje}
TODO

\section{dim = 20}
\subsection{Wykresy Alpine01}
\subsubsection{Pure-Random-Search}
TODO

\subsubsection{Multi-Start}
TODO

\subsubsection{Obserwacje}
TODO

\subsection{Wykresy funkcji Ackley’a}
\subsubsection{Pure-Random-Search}
TODO

\subsubsection{Multi-Start}
TODO

\subsubsection{Obserwacje}
TODO

\newpage
\part{Podsumowanie}
\section{Podsumowanie}
TODO

\newpage
\section{Bibliografia}
Źródła i inspiracje wykorzystane przy tworzeniu projektu:
\begin{itemize}
  \item Wykłady z Rachunku prawdopodobieństwa i statystyki, prowadzone przez dr hab. Macieja Smółkę, na 3 semestrze Informatyki AGH WI.
  \item \url{https://www.agh.edu.pl/o-agh/multimedia/znak-graficzny-agh/}
\end{itemize}





\part{PRZYKŁADY}
\section{jak-zrobic-kod}
\begin{lstlisting}[
  style=vscode-darkplus-python,
  caption={Konstruktor klasy KdTree\_Visualizer.},
  label=lst:python_code,
  captionpos=b
]
from kdtree import KdTree_Visualizer

points1 = [[1, 2], [3, 4], [5, 6]]
# w folderze zostanie utworzony plik kdtree.gif
Ktree1 = KdTree_Visualizer(points1, title='KdTree', filename='kdtree')
# w folderze zostanie utworzony plik kdtree-construction.gif
Ktree2 = KdTree_Visualizer(points1)
\end{lstlisting}

\section{jak-zrobic-foto}
\begin{figure}[H]
  \centering
  \includegraphics[width=0.5\textwidth]{resources/dimensions_graph.png}
  \caption{Wykres porównujący wydajność KdTree dla różnych ilości wymiarów.}
  \label{fig:dimensions_graph}
\end{figure}

\end{document}